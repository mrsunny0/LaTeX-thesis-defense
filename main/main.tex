%---------------------------------------------------------------------------%
%                                 PREAMBLE                                  %
%---------------------------------------------------------------------------%
% declare document class, 12pt, lettersize and article 
% could also be report, however section headers turn into chapters
% \documentclass[12pt, lettersize]{article}
\documentclass[12pt,twoside]{../main/mitthesis}

% use subfiles to modularize pieces of the LaTeX code
\usepackage{subfiles}

% import preamble.sty for packages
% import refextdoc.sty for subfile crossreferencing
% note the relative import. Because subfiles (e.g. abstract, introduction, etc.)
% are located in separate files, for all files to be obtainable by main and subfiles,
% need to tranverse up to the root directory (../) then back to the appropriate files in /main
\usepackage{../main/_preamble}
\usepackage{../main/_refextdoc}

% relative imports of images
% note that graphics path should be encased in {}, ends with / to define directory, 
% and finally separated by , without any lead or lag spaces.
% therefore graphicsspath should look like \graphiscpath{{abc/},{xyz/},{123/}}
% spaces in directory names are not recommended, however the the \usepackage[space]{grffile}
% will attempt to work with directories with spaces
\graphicspath{{../chapter_1/_figures/},
				{../chapter_2/_figures/},
				{../chapter_3/_figures/},
				{../chapter_4/_figures/},
				{../chapter_5/_figures/},
				{../appendix/_figures/}}
				
% relative imports for \input
% typically used for tables which are input'd per chapter/appendix
% names of input files can be written as is without the relative path prefix
\makeatletter 
	\def\input@path{{../chapter_1/_tables/},
					{../chapter_2/_tables/},
					{../chapter_3/_tables/},
					{../chapter_4/_tables/},
					{../chapter_5/_tables/},
					{../appendix/_tables/}}
\makeatother
				
% set bibliographies
\usepackage[citestyle=numeric,
	style=numeric-comp, % cite instances such as 1-5
	sorting=none,
	refsection=chapter,
	doi=false,	% supress digital object identifier
	isbn=false,	% supress international standard book number
	url=false	% suppress url and date of access
	]{biblatex}		
		
% establish bibliograhies paths	
\addbibresource{../chapter_1/_bib/bib_chapter1.bib}
%\addbibresource{../chapter_2/_bib/bib_chapter2.bib}	
%\addbibresource{../chapter_3/_bib/bib_chapter3.bib}	
%\addbibresource{../chapter_4/_bib/bib_chapter4.bib}	
%\addbibresource{../chapter_5/_bib/bib_chapter5.bib}		
%\addbibresource{../appendix/_bib/bib_appendix.bib}

%% MIT specific packages 
% These have been added at the request of the MIT Libraries, because
% some PDF conversions mess up the ligatures.  -LB, 1/22/2014
\usepackage{cmap}
\usepackage[T1]{fontenc}
\pagestyle{plain}

%% Personal preference
% prevent hyphenation of specific words
% \hyphenation{thatshouldnot}
% remove all hyphenations
\global\hyphenpenalty=100000

%---------------------------------------------------------------------------%
%                                BEGIN DOCUMENT                             %
%---------------------------------------------------------------------------%
\begin{document}

%------------------%
% FRONT MATTER
%------------------%
% Title page
\title{THESIS TITLE}

\author{FIRST MIDDLE LASTNAME}

% Previous degrees
\prevdegrees{B.S., SOME UNIVERSITY (YEAR)}
%       \prevdegrees{B.S., University of California (1978) \\
%                    S.M., Massachusetts Institute of Technology (1981)}

% Current degree
\department{Department of Biological Engineering}
\degree{Doctor of Philosophy in Biological Engineering}
% \degree{Doctor of Philosophy \and Master of Science}

% Date of degree
\degreemonth{MONTH}
\degreeyear{YEAR}
\thesisdate{MONTH DAY, YEAR}

%% By default, the thesis will be copyrighted to MIT.  If you need to copyright
%% the thesis to yourself, just specify the `vi' documentclass option.  If for
%% some reason you want to exactly specify the copyright notice text, you can
%% use the \copyrightnoticetext command.  
%\copyrightnoticetext{\copyright Myself, 2019.  Do not open until 2050.}

% Supervisors
\supervisor{SUPERVISOR}{SUPERVISOR TITLE}

% This is the department committee chairman, not the thesis committee
% chairman.  You should replace this with your Department's Committee
% Chairman.
\chairman{CHAIR}{Chair of Graduate Program, Department of Biological Engineering}

% Make the titlepage based on the above information.  If you need
% something special and can't use the standard form, you can specify
% the exact text of the titlepage yourself.  Put it in a titlepage
% environment and leave blank lines where you want vertical space.
% The spaces will be adjusted to fill the entire page.  The dotted
% lines for the signatures are made with the \signature command.
\maketitle

%\cleardoublepage

% Abstract
\pagestyle{empty}
% \setcounter{page}{\thesavepage}
\setcounter{savepage}{\thepage}
\begin{abstractpage}
ABSTRACT
\end{abstractpage}
%\cleardoublepage

% Acknowledgements
\pagestyle{empty}
\section*{Acknowledgments}
ACKNOWLEDGMENT 
%\cleardoublepage

% Signatures (optional)
\pagestyle{empty}
\begin{titlepage}
\begin{large}
This doctoral thesis has been examined by a Committee of the Department
of Biological Engineering as follows:

\signature{Professor K. Dane Wittrup}{Chairman, Thesis Committee \\
   Professor of Biological and Chemical Engineering}

\signature{Professor Angela M. Belcher}{Thesis Supervisor \\
   Professor of Biological Engineering and Material Science}

\signature{Professor Cathy L. Drennan}{Member, Thesis Committee \\
   Professor of Biology and Chemistry}
\end{large}
\end{titlepage}


%\cleardoublepage

%------------------%
% TABLE OF CONTENTS
%------------------%
% some thesis documents wil also want a table of contents for figures and tables
% uncomment the commands below to create a table of contents for figures and tables
\tableofcontents
%\newpage
\listoffigures
%\newpage
\listoftables
 
% page numbering can be roman (e.g. i, ii, iii, iv), alpha (a, b, c, d)
% or arabic (1,2,3,4). Change the below option to roman for front matter text
% such as preface pages, alph for alpha lettering (e.g. Appendix maybe)
% or arabic for standard page numbering. If you want to restart the page 
% counter because you are using a new numbering format, you can use 
% \setcounter{page}{X} where X is the new number you want to count on.
\pagenumbering{arabic}
\pagestyle{plain}

%---------------------------------------------------------------------------%
%                                  CHAPTERS                                 %
%---------------------------------------------------------------------------%
% Chapter 1 - 
\subfile{../chapter_1/chapter_1}

% Chapter 2 - 
\subfile{../chapter_2/chapter_2}

% Chapter 3 - 
\subfile{../chapter_3/chapter_3}

% Chapter 4 - 
\subfile{../chapter_4/chapter_4}

% Chapter 5 - 
\subfile{../chapter_5/chapter_5}

%---------------------------------------------------------------------------%
%                                  APPENDIX                                 %
%---------------------------------------------------------------------------%
\newpage
\appendix

% A
\subfile{../appendix/appendix_A}

% B 
\subfile{../appendix/appendix_B}

% C 
\subfile{../appendix/appendix_C}

%---------------------------------------------------------------------------%
%                                 BIBLIOGRAPHY                              %
%---------------------------------------------------------------------------%
% \begin{singlespace}
% \bibliography{../bib/bib_chapter1}
% \bibliographystyle{plain}
% \end{singlespace}

%---------------------------------------------------------------------------%
%                                 END DOCUMENT                              %
%---------------------------------------------------------------------------%
\end{document}